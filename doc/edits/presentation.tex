\documentclass{beamer}

% ---- PACKAGES ----
\usepackage[francais]{babel}
\usepackage[T1]{fontenc}
\usepackage[utf8]{inputenc}

% ---- THEME ----
\usetheme{Dresden}
\usecolortheme{beaver}

% ---- PAGE NUMBER ----
\setbeamertemplate{footline}%{miniframes theme}
{%
	\begin{beamercolorbox}[colsep=1.5pt]{upper separation line foot}
	\end{beamercolorbox}
	\begin{beamercolorbox}[ht=2.5ex,dp=1.125ex,%
		leftskip=.3cm,rightskip=.3cm plus1fil]{author in head/foot}%
		\leavevmode{\usebeamerfont{author in head/foot}\insertshortauthor}%
		\hfill%
		{\usebeamerfont{institute in head/foot}\usebeamercolor[fg]{institute in head/foot}\insertshortinstitute}%
	\end{beamercolorbox}%
	\begin{beamercolorbox}[ht=2.5ex,dp=1.125ex,%
		leftskip=.3cm,rightskip=.3cm plus1fil]{title in head/foot}%
		{\usebeamerfont{title in head/foot}\insertshorttitle} \hfill     \insertframenumber%
	\end{beamercolorbox}%
	\begin{beamercolorbox}[colsep=1.5pt]{lower separation line foot}
	\end{beamercolorbox}
}

% ---- TITLE PAGE DECLARATION ----
\title[Hel : The pixelated horror]{Hel : The pixelated horror}
\subtitle{Travail de diplôme 2015}
\author{Yannick R. Brodard}
\institute{Centre de Formation Professionnelle Technique\\École d'informatique}
\date{Vendredi, 5 juin 2015}
\subject{Computer Science}

% ---- DOCUMENT BEGINNING ----
\begin{document}
\frame{\titlepage}

% ---- TABLE OF CONTENTS
\begin{frame}
\frametitle{Sommaire}
\tableofcontents[hideallsubsections]
\end{frame}

\AtBeginSection[]
{
  \begin{frame}
    \frametitle{Table of Contents}
    \tableofcontents[currentsection,hideallsubsections]
  \end{frame}
}

% ---- CONTEXTE DU PROJET ----
\section{Contexte du projet}
\subsection{Cahier des charges}
\begin{frame}
\frametitle{Sujet}
\begin{itemize}
	\item Jeu-vidéo 2D
	\item Aspects RPG
	\begin{itemize}
		\item Système de compétences
		\item Système d'objets
	\end{itemize}
	\item Monogame
\end{itemize}
\end{frame}

\begin{frame}
\frametitle{But}
\begin{itemize}
	\item Gagner des compétences
	\item Trouver des objets pour améliorer son personnage
	\item Parcourir un monde qui n'est jamais deux fois le même
	\item Vaincre les ennemis et le boss \textit{Hel}
\end{itemize}
\end{frame}

\begin{frame}
\frametitle{Détails}
\begin{itemize}
	\item Le joueur peut :
	\begin{itemize}
		\item Se déplacer
		\item Attaquer
		\item Invoquer des sorts
	\end{itemize}
	\item La carte est générée aléatoirement
	\item Un système de progression du personnage est mis en place
	\begin{itemize}
		\item Points de caractéristiques
		\item Points de compétences
	\end{itemize}
	\item Des objets sont trouvables pour le joueur
\end{itemize}
\end{frame}

\subsection{Étude du marché}
\begin{frame}
\frametitle{Le marché actuel}
\begin{itemize}
	\item Diablo III, poids lourd actuel
	\item Path of Exile, concurrent de Diablo III
	\item Bastion, jeu indépendant remarquable
\end{itemize}
\end{frame}

\begin{frame}
\frametitle{Adaptation au marché}
\begin{itemize}
	\item Rendre le jeu équilibré pour satisfaire les hardcores-gamers ainsi qu'introduire des débutants.
	\item Rendre le jeu générique pour facilement ajouté du contenu. Ce qui permet un développement plus solide de la communauté.
	\item \textit{Rendre le jeu graphiquement attractif}
\end{itemize}
\end{frame}

% ---- FONCTIONNALITÉS ----
\section{Fonctionnalités}
\subsection{}
\begin{frame}
\frametitle{Interface utilisateur}
\begin{itemize}
	\item Menu de navigation
	\begin{itemize}
		\item Jouer
		\item Options
		\item Charger
	\end{itemize}
	\item Écran de présentation
	\item Écran de jeu
\end{itemize}
\end{frame}

\begin{frame}
\frametitle{Personnage}
\begin{itemize}
	\item Possède des caractéristiques
	\item Peut porter des objets pour augmenter ses caractéristiques
	\item Peut stocker des objets
\end{itemize}
\end{frame}

\begin{frame}
\frametitle{Caractéristiques \& compétences}
\begin{itemize}
	\item Ce qui défini un personnage du jeu
	\item Tous les actions dépendent des caractéristiques
	\item Les personnages possèdent des caractéristiques par défauts
	\item Les objets ont des caractéristiques qui augmente ceux du personnage jouable
	\item Certains sorts ont des caractéristiques qui augmente ceux du personnage pendant un certain temps
\end{itemize}
\end{frame}

\begin{frame}
\frametitle{Objets}
\begin{itemize}
	\item Peuvent être trouvé en...
	\begin{itemize}
		\item ...tuant un ennemi
		\item ...ouvrant un coffre
	\end{itemize}
	\item Augmentent les caractéristiques du personnage
	\item Se placent sur le personnage
	\begin{itemize}
		\item tête, épaules, torse
		\item jambes, pieds, mains
		\item collier, bagues
	\end{itemize}
\end{itemize}
\end{frame}

\begin{frame}
\frametitle{Carte}
\begin{itemize}
	\item Générés aléatoirement
	\item Différentes difficultés
\end{itemize}
\end{frame}


% ---- MÉTHODES MISES EN OEUVRE ----
\section{Méthodes mises en œuvre}
\subsection{Généralités}
\begin{frame}
\frametitle{Les boucles de Monogame}
\end{frame}

\begin{frame}
\frametitle{Structure du projet}
\end{frame}

\begin{frame}
\frametitle{Génération de la carte}
\end{frame}

% ---- ÉLÉMENTS DU JEU ----
\subsection{Éléments du jeu}
\begin{frame}
\frametitle{Caractéristiques}
\end{frame}

\begin{frame}
\frametitle{Entités}
\end{frame}

\begin{frame}
\frametitle{Déplacements}
\end{frame}

\begin{frame}
\frametitle{Ennemis}
\end{frame}

\begin{frame}
\frametitle{Attaques}
\end{frame}

% ---- VUE ----
\subsection{Vue}
\begin{frame}
\frametitle{Gestion de l'écran}
\end{frame}

\begin{frame}
\frametitle{Les différents écrans}
\end{frame}

\begin{frame}
\frametitle{Gestion des textures}
\end{frame}

% ---- OUTILS ----
\subsection{Outils}
\begin{frame}
\frametitle{Sérialisation XML}
\end{frame}

\begin{frame}
\frametitle{Intersections}
\end{frame}

\begin{frame}
\frametitle{Gestion des entrées de prériphériques}
\end{frame}

\begin{frame}
\frametitle{Primitives 2D}
\end{frame}
% ---- DÉMONSTRATION ----
\section{Démonstration}
\subsection{}
\begin{frame}
\frametitle{Démonstration}
\end{frame}

% ---- CONCLUSION ET PERPECTIVES ----
\section{Conclusion et perspectives}
\subsection{}
\begin{frame}
\frametitle{Apport personnel}
\end{frame}

\begin{frame}
\frametitle{Conclusion et bilan personnel}
\end{frame}

\begin{frame}
\frametitle{Perspectives du projet}
\end{frame}

\begin{frame}
\frametitle{Questions ?}
\end{frame}

\end{document}